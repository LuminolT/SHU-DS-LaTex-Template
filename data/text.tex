\chapter{实验目的与要求}

\chapter{实验环境}

本次实验的实验环境如下表\ref{tab:env}所示。

\begin{table}[!htbp]
    \centering
    \caption{实验环境}
    \label{tab:env}
    \begin{tabular}{ll}
    \toprule[1pt]  
    开发环境       & CLion 2021.2  \\
    开发语言及其标准    & C++14         \\
    编译器         & MinGW          \\
    \bottomrule[1pt]
    \end{tabular}
\end{table}

\chapter{实验内容}

\begin{enumerate}
    \item A
    \item B
    \item C
\end{enumerate}


\chapter{实验内容的设计与实现}

[说明:选择本实验中最有程序设计技巧或特色的并具有独立功能的源代码片断,并对其算法的技巧或特色进行必要的文字说明. ]

本实验所用算法如下算法\ref{alg:1}所示。

\begin{algorithm}
    \caption{这是一个简单的算法}
    \label{alg:1}
    \begin{algorithmic}[1]
        \REQUIRE    整数$a$
        \ENSURE     整数$b$
        \STATE      $b := a$
        \RETURN     $b$
    \end{algorithmic}
\end{algorithm}

本实验部分代码如下代码\ref{code:1}所示。

\begin{lstlisting}[language = C++, caption = {这是一段简单的代码}, label = {code:1}]
#include <iostream>
using namespace std;
int main() {
    int a = 0;
    int b = a;
    return 0;
}
\end{lstlisting}

\chapter{收获与体会}

[说明:每位组员撰写完成该实验后的收获和体会。]